\chapter{Proposed Operators for a Parallel Implementation of J} 
\label{paraop}

\section{Rationale}
Ideally, a parallel library would automatically exploit potential concurrency, 
and thus reduce a program's runtime, 
without requiring the programmer to write a single line of parallelizing code. 
However, this is often not the case,  
because frequently programmers must use domain knowledge 
of the problem or platform to achieve good run-time results. 

To grant this flexibility in future parallel implementations of J, 
this section introduces two new operators, called the \textit{parallel rank operator} 
and the \textit{parallel insert operator}.
The \textit{parallel rank operator}, described in Section \ref{prank}, 
would allow the programmer to specify the data ranks on which to parallelize code.
The \textit{parallel insert operator}, described in Section \ref{pins}, 
would allow the programmer to parallelize reducing operations with associative functions.
Also included is a new system library 
that would allow the programmer to change the underlying parallel environment,
described in Section \ref{pfor}

The spellings \texttt{"::} and \texttt{/::} for the two parallel operators 
require no changes to the existing J lexer\cite{ioj}.
Fig~\ref{fig:jlexer-pres} demonstrates this, showing that the strings 
\texttt{"::} and \texttt{\::} are whole lexemes and so are placed entirely in their own boxes, 
instead of being split between several boxes.
Also, the spellings use the same base character as their sequential analogs, (\texttt{"} and \texttt{/}), 
as well as two ``parallel'' colons, making it easier for the programmer to remember their functions.

\begin{figure}[htbp]
\begin{quote}
\begin{singlespacing}
\begin{small}
\framebox[\linewidth][l]{
\BVerbatimInput{jlexer-preserved.txt}
}
\end{small}
\end{singlespacing}
\end{quote}
\caption{The current J lexer recognizes \texttt{"::} and \texttt{/::} as lexemes}
\label{fig:jlexer-pres}
\end{figure}

For the parallel environment library, 
it seems best to augment the existing ``\gls{foreign}'' operator (spelled \texttt{!:}).  
The foreign operator allows
the programmer to change environmental parameters 
such as print precision, file I/O, etc\cite{jvocab}, 
and so fits the idea of changing the state of a parallel environment. 
\texttt{111} is a suitable numeric encoding for the parallel library because 
it lies within the range of used values, is not already in use, and 
mnemonically it looks like three parallel lines (\texttt{11} was already taken)\cite{jvocab}.

\section{Parallel Rank Operator: \texttt{"::}}
\label{prank}

\subsection{Usage}
The proposed parallel rank operator \texttt{"::} 
is a two-argument higher order function (conjunction)
whose first argument is a function 
and whose second argument is the ranks to both apply and parallelize the function. 
Its output would be functionally equivalent to the (sequential) rank operator, i.e. 
\[\texttt{f (":: r) y} \Leftrightarrow \texttt{f (" r) y}\] and
\[\texttt{x f (":: r) y} \Leftrightarrow \texttt{x f (" r) y}\]

\noindent for all $f, r, s$, and $y$.
\texttt{"::} would override defaults for automatic parallelization 
to guarantee the resulting function creates parallelizable tasks 
on the specified cells of its argument(s).
These tasks would then be distributed to the available threads.

To illustrate, consider the examples in Fig~\ref{fig::pr_tasks}. 
Fundamentally \texttt{increment} always operates on numeric scalars, 
so it always yields the same result regardless of cells to which it is applied. 
This property would also hold when using the parallel rank operator. 
However, while the results of \texttt{increment} would be independent of rank, 
the tasks created by \texttt{"::} would not be.
As Fig~\ref{fig::pr_tasks} shows, 
the line \texttt{increment (":: 1)} would create tasks 
for incrementing each of the vector elements of \texttt{mat2\_3} in parallel.
In general, the programmer could create parallelizable tasks from incrementing
the six scalars, the two vectors, or the one matrix of \texttt{mat2\_3}.

\pagebreak

\glsadd{increment}
\glsadd{showTasks}

\begin{figure}[hptb]
\begin{quote}
\begin{singlespacing}
\begin{small}
\framebox[\linewidth][l]{
\BVerbatimInput{showTasks-sing.txt}
}
\end{small}
\end{singlespacing}
\end{quote}
\caption{Visualizing tasks created by the parallel rank operator}
\label{fig::pr_tasks}
\end{figure}

Fig~\ref{fig::pr_tasks2} gives a more complex example of how \texttt{"::} would create parallel tasks  
involving functions of two arguments as well as repeated applications of the rank operator,
Using the parallel rank operator, the line of code
\texttt{mat2\_3 +("::2) arr2\_2\_3} would create parallelizable tasks for adding \texttt{mat2\_3} to 
each of the matrix (rank 2) items of \texttt{arr2\_2\_3}, 
whereas \texttt{mat2\_3 +("::1)("2) arr2\_2\_3} would parallelize addition 
of each of the vectors in both arrays.

\begin{figure}[p]
\begin{quote}
\begin{singlespacing}
\begin{small}
\framebox[\linewidth][l]{
\BVerbatimInput{showTasks-doub.txt}
}
\end{small}
\end{singlespacing}
\end{quote}
\caption{Visualizing tasks created by the parallel rank operator for two collections}
\label{fig::pr_tasks2}
\end{figure}

Some open questions remain, notably 
how the parallel rank operator should behave when used multiple times in the same function.
Should multiple applications of \texttt{"::} create multiple levels of parallelizable tasks, 
increasing the overhead of creating and scheduling threads? 
Or should only the outermost occurrence determine the parallel tasks, 
unparallelizing programs when they are used in larger functions?
Either possibility could significantly impact the program's performance.
One solution might be to let programmers chose which behavior best suites their needs 
using the parallel environment library described in Section \ref{pfor}.

\pagebreak

\section{Parallel Insert Operator: \texttt{/::}}
\label{pins}
\subsection{Rationale}
Programmers often need need to perform some reducing operation on large collections, 
for example finding the sum, product, maximum, minimum, etc. of a collection of numbers.
When the reducing operation $f$ is associative, 
i.e. when $f(x,y) = f(y,x)$ for all $f, x, y$ 
(and when dealing with floating point values, for a sufficiently tolerant comparison \texttt{=}),
then the reduction can be executed in parallel with little effort on the programmer's part.
Therefore, a future parallel implementation of J should allow the programmer 
to explicitly reduce an array with an associative function in parallel, 
since J's (sequential) insert operator does not assume its argument function is associative, 
nor is it immediately obvious how to determine if a user-defined function is associative.

\subsection{Usage}
The proposed parallel insert operator \texttt{/::} is a single-argument higher order function (adverb) 
whose argument \texttt{u} is an associative two-argument function 
and which returns a function that reduces a collection with \texttt{u}.
The (sequential) insert operator and \texttt{\::} would be 
functionally equivalent for all associative functions.
Thus, the J session depicted in Fig~\ref{fig::pinsert} would have the same results for every line of code executed, 
regardless of whether \texttt{insert} was assigned to be \texttt{/::} or \texttt{/}

\glsadd{insert}

\begin{figure}[h]
\begin{quote}
\begin{singlespacing}
\begin{small}
\framebox[\linewidth][l]{
\BVerbatimInput{par-insert.txt}
}
\end{small}
\end{singlespacing}
\end{quote}
\caption{Using the parallel insert operator}
\label{fig::pinsert}
\end{figure}

Some open questions remain, notably 
how the parallel insert operator should behave on non-associative functions. 
Without associativity, the result of reduction depends on the order the function is applied to an array's items, 
thus destroying any possibility for parallelism.
One way to deal with non-associativity is to restrict the domain of \texttt{/::} to associative primitives. 
This prevents logical errors such as radically different results for the same calculations on the same data, 
depending on the thread scheduling scheme. 
However, this approach would likely require adding an additional table of information 
storing whether the given primitive is associative.
This approach would also exclude user-defined associative functions 
from benefiting from parallelism.

Alternatively, \texttt{/::} might ignore the question of associativity 
and try to parallelize all of its functions as if they were associative.
This approach could lead to errors difficult for a novice parallel programmer to debug, 
since they would occur at the logic level. 
Finally, the decision might be left to programmers by allowing them to switch between either of these options 
using the parallel environment library, discussed in Section \ref{pfor}.

\section{Parallel Environment Library: \texttt{111 !:}} 
\label{pfor}

\subsection{Conventions of the Foreign Operator}
The foreign operator is a two-argument higher order function, or in J a \textit{conjunction}, 
whose arguments are always numeric.
Conceptually, these arguments serve as an index into system libraries and functions.
The left argument indexes the desired library.
E.g., $2$ indexes the library for functions which affect the host machine, 
$9$ indexes the library for viewing and setting global J parameters (such as print precision), and etc.
The second argument indexes a specific function.
For example, \texttt{9!:6}  is the function which displays the print characters for J's box type.

\subsection{Usage}
There are many options to consider for implementing a full suite of functions for a parallel environment library.
A more full review of the functionality of other shared-memory parallel libraries, such as OpenMP\cite{openmp}, 
is required to make sure such a library provides all the functionality parallel programmers expect.
However, two types of functionality would almost certainly be required.
They are:

\begin{enumerate}
    \item Getting and setting the total number of threads available in the parallel environment,
        with an extension for J's value for infinity to mean no user-specified limit on the number of threads, and
    \item Getting and setting the thread scheduling schemes (such as static, round-robin, etc.).
        These would be encoded as numeric values, in keeping with the conventions of the foreign libraries.
\end{enumerate}

\noindent Additionally, as discussed in Sections \ref{pfor} and \ref{pins}, it may be desirable to change 
the default behavior of the parallel rank operator and/or parallel insert operator, 
to parallelize on user-defined associative operators.

\begin{figure}[h]
\begin{quote}
\begin{singlespacing}
\begin{small}
\framebox[\linewidth][l]{
\BVerbatimInput{par-foreign.txt}
}
\end{small}
\end{singlespacing}
\end{quote}
\caption{A hypothetical J session using the parallel environment library}
\label{fig::pfor}
\end{figure}

The example given in Fig~\ref{fig::pfor} is not based on any existing J REPL, 
but an illustration of how programmers might use this proposed library.
The last line would parallelize addition on the row elements of \texttt{mat2\_3} and \texttt{arr2\_2\_3},
then parallelize the sum operation on each of the vectors of the resulting array, 
using 4 threads with a round-robin scheduling scheme.
